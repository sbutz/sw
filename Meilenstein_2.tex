\documentclass[12pt, a4paper, titlepage]{article}

\usepackage{geometry}
\geometry{left=2.5cm, right=2.5cm, top=2.5cm, bottom=3cm}
\usepackage[utf8]{inputenc}
\usepackage[ngerman]{babel}
\usepackage{graphicx}
\usepackage{float}
\usepackage{parskip}
\usepackage[dvipsnames]{xcolor}
\usepackage[colorlinks, citecolor=black, filecolor=black, linkcolor=black,       
	urlcolor=black]{hyperref} 

\usepackage{booktabs}

\newcommand{\andi}[0]{Andreas Huber}
\newcommand{\andiProj}[0]{\textit{ynvest}}
\newcommand{\sina}[0]{Sina Amann}
\newcommand{\sinaProj}[0]{\textit{eBank}}


\begin{document}

\title{2. Meilenstein im Fach Softwareentwicklung}
\author{Stefan Butz \\
		Studiengruppen: IN7 \\
		Matrikelnummer: 3175600}

\maketitle

\section{Einleitung}
Dieses Dokument enthält die aktualisierten Bausteine des ersten Meilensteins.
Um eine Mischung aus deutschen und englischen Bezeichnern im Quellcode zu
vermeiden, wurden alle Bezeichner ins Englische übersetzt.
Aus diesem Grund wird jedes Diagramm beigelegt, auch falls es darin sonst
keine funktionalen Änderungen gab.

Folgende Farblegen gibt Hilfestellung beim Lesen der nachfolgenden Diagramme.
\begin{itemize}
	\item \textcolor{green}{grün}:
	Gekennzeichnetes Objekt wurde gegenüber dem ersten Meilenstein hinzugefügt.
	\item \textcolor{YellowOrange}{orange}:
	Gekennzeichnetes Objekt wurde gegenüber dem ersten Meilenstein geändert.
	\item \textcolor{red}{rot}:
	Gekennzeichnetes Objekt wurde gegenüber dem ersten Meilenstein entfernt.
	\item \textcolor{blue}{blau}:
	Gekennzeichnetes Objekt ist Teil eines Partnerprojektes.
\end{itemize}

\clearpage
\section{Anwendungsfalldiagramm}
\begin{figure}[H]
	\centering
    %\includegraphics[width=\textwidth]{Use-Case-Diagramm.png}
    \includegraphics[height=15.5cm]{Use-Case-Diagramm.png}
	\caption{Use-Case-Diagramm mit wesentlichen Anwendungsfällen}
	\label{fig:use_case}
\end{figure}

\clearpage
\section{Domänenmodell}
Die größte Änderung betrifft hier die Entität User.
Durch den neuen Themnblock Web Security kamen bis dahin unbekannte
Anforderungen an die Benutzer-Verwaltung hinzu. Aus diesem Grund wurden
die Entitäten Staff und TradingPartner durch entfernt und ein Rollensystem
hinzugefügt.

\begin{figure}[H]
	\centering
    \includegraphics[width=\textwidth]{Domaenenmodell.png}
    %\includegraphics[width=17cm]{Domaenenmodell.png}
	\caption{Domänenmodell als Klassendiagramm}
	\label{fig:domain}
\end{figure}

Bei den Beziehungen zwischen User und Order wurde absichtlich auf eine
bidirektionale Beziehung verzichtet. Dafür gibt es zwei Gründe.
\begin{enumerate}
	\item Ein Nutzer (möglicherweise ein Handelspartner, der wiederum sehr
	viele Kunden hat) erstellt potenziel viele Aufträge.
	Diese in eine Java Collection zu laden (auch mit Lazy Loading) wäre
	ineffizient.
	\item Das Abfragen der Aufträge aus einem Repository erlaubt einfachere
	und effizientere Filterung, Sortierung und Paginierung als der Zugriff
	über eine Java Collection.
\end{enumerate}
Gleiches gilt für die Beziehungen zwischen Share und Order sowie Share und
Transaction.

\section{Komponentenmodel}
\begin{figure}[H]
	\centering
    \includegraphics[width=\textwidth]{Komponentendiagramm.png}
	\caption{Systemarchitektur als Komponentenmodell}
	\label{fig:components}
\end{figure}

\section{Schnittstellen}

\subsection{Angebotene Schnittstellen}
\label{subsec:Schnittstellen}
\begin{figure}[H]
	\centering
    \includegraphics[width=\textwidth]{Schnittstellen.png}
	\caption{Schnittstellenbeschreibung als Interfaces}
	\label{fig:interfaces}
\end{figure}
Die in \autoref{fig:interfaces} abgebildeten Schnittstellen werden von den
Partnerprojekten \sinaProj von \sina und \andiProj von \andi genutzt.

\subsection{Konsumierte Schnittstellen}
Vom Partnerprojekt \sinaProj wird der Überweisungdienst genutzt. Eine genaue
Schnittstellenbeschreibung dessen entnehmen Sie bitte den Planungsdokumenten
von \sina.


\section{Anmerkungen zur fachlichen Implementierung}
Auf eine Speicherung, welcher Kunde welche Aktien besitzt, wurde bewusst
verzichtet. Die Verantwortung dafür liege bei einem gesonderten Projekt,
einem elektronischen Wertpapierregister. Die Bunderregierung von Deutschland
erarbeitet aktuell die Anforderung, die ein solches erfüllen müsste.
Eine sinnvolle Implementierung ist daher nur bedingt möglich.

\section{Anmerkungen zur technischen Implementierung}
Folgende technischen Besonderheiten verdienen eine Erwähnung:
\begin{itemize}
	\item Zur Ansteuerung der Partnerschnittstellen wurde nicht die Klasse
	RestTemplate sondern WebClient verwendet. RestTemplate ist deprecated.
	WebClient ist der offizielle Nachfolger. WebClient ermöglicht ausßerdem
	einfacher Authenfizierung über HTTP Basic Auth.
	\item Um einen flüssigen Handel zu bieten und die Anwendung besser zu
	veranschaulichen, wurde ein Bot implementiert. Dieser stellt in regelmäßigen
	Abständen neue Angebot und Nachfragen ein und befriedigt so den Markt.
	\item Durch die Nutzung vieler Repositories und die Injektion des aktuellen
	Nutzer in die Controller wurde stets der Scope Singleton verwendet.
	Obwhol dies der Standard für alle Beans ist, wurde, um die Requirements zu
	erfüllen, jede Bean mit der Annotation Scope gekennzeichnet.
	\item Es wurde ein Testnutzer mit dem Nutzernamen admin und dem
	korrektorfreundlichen Password 123 angelegt.
\end{itemize}

\end{document}