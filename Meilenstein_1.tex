\documentclass[12pt, a4paper, titlepage]{article}

\usepackage{geometry}
\geometry{left=2.5cm, right=2.5cm, top=2.5cm, bottom=3cm}
\usepackage[utf8]{inputenc}
\usepackage[ngerman]{babel}
\usepackage{graphicx}
\usepackage{float}
\usepackage{parskip}
\usepackage[colorlinks, citecolor=black, filecolor=black, linkcolor=black,       
	urlcolor=black]{hyperref} 

\usepackage{booktabs}


\begin{document}

\title{1. Meilenstein im Fach Softwareentwicklung}
\author{Stefan Butz \\
		Studiengruppen: IN7 \\
		Matrikelnummer: 3175600}

\maketitle

\section{TODO}
\begin{itemize}
	\item Ist die Darstellung der externen System so in Ordnung?
	\item Modell mit highlighted Änderungen darstellen
\end{itemize}

\section{Einleitung}
Dieses Dokument enthält die geforderten Bausteine für den ersten Meilenstein
im Fach Softwareentwicklung.
Als Softwareprojekt soll eine webbasierte Handelsplattform für Wertpapiere
realisiert werden.
Schnittstellen gibt es zu den Partnerprojekten \textit{Amann Banking} von Sina
Amann und \textit{investy} von Andreas Huber.
Eine genaue Beschreibung der Schnittstellen finden sich in
\autoref{subsec:Schnittstellen}.

\section{Anwendungsfälle}
\label{sec:use_case}
\subsection{Anwendungsfalldiagramm}
Die wesentlichen Funktionen, welche die Handelsplattform  bieten soll, werden in
\autoref{fig:use_case} dargestellt und im Folgenden genauer erläutert.

\begin{figure}[H]
	\centering
    \includegraphics[width=\textwidth]{Use-Case-Diagramm.png}
	\caption{Use-Case-Diagramm mit wesentlichen Anwendungsfällen}
	\label{fig:use_case}
\end{figure}

\subsection{Use-Case-Beschreibungen}
\begin{tabular}{|p{.30\textwidth}|p{.70\textwidth}|}
	\hline
	Use-Case & Wertpapiere abfragen \\
	\hline
	Vorbedingungen & - \\
	\hline
	Wesentliche Schritte &
		\begin{itemize}
			\item Auflistung der Wertpapiere oder Kursabfrage zu einem Wertpapier
		\end{itemize} \\
	\hline
	Nachbedingungen & - \\
	\hline
\end{tabular}\par

\begin{tabular}{|p{.30\textwidth}|p{.70\textwidth}|}
	\hline
	Use-Case & Auftrag erstellen \\
	\hline
	Vorbedingungen &
		\begin{itemize}
			\item Benutzer wurde angelegt
			\item Wertpapier, dass gehandelt werden soll, wurde angelegt
		\end{itemize} \\
	\hline
	Wesentliche Schritte &
		\begin{itemize}
			\item Mit Namen und Passwort anmelden
			\item Auftrag ausfüllen (Welches Wertpapier soll in welcher Anzahl und zu
			welchem Maximalpreis gekauft bzw. zu welchem Minimalpreis verkauft werden?)
			\item Auftrag abschicken
			\item Auftragsbestätigung erhalten
			\item Status des Auftrages mithilfe der Auftragsnummer abfragen
		\end{itemize} \\
	\hline
	Nachbedingungen &
		\begin{itemize}
			\item Auftrag wird ausgeführt sobald ein passender Gegenauftrag gefunden
			wird
			\item Geld wird auf das angegebene Konto überwiesen bzw. vom angegeben Konto
			eingezogen
		\end{itemize} \\
	\hline
\end{tabular}\par

\begin{tabular}{|p{.30\textwidth}|p{.70\textwidth}|}
	\hline
	Use-Case & Auftrag stornieren \\
	\hline
	Vorbedingungen &
		\begin{itemize}
			\item Benutzer wurde angelegt
			\item Benutzer hat Auftrag angelegt
			\item Auftrag wurde noch nicht ausgeführt
		\end{itemize} \\
	\hline
	Wesentliche Schritte &
		\begin{itemize}
			\item Mit Namen und Passwort anmelden
			\item Auftrag mithilfe der Auftragsnummer stornieren
			\item Stornierungsbestätigung erhalten
		\end{itemize} \\
	\hline
	Nachbedingungen &
		\begin{itemize}
			\item Auftrag ist storniert
		\end{itemize} \\
	\hline
\end{tabular}\par

\begin{tabular}{|p{.30\textwidth}|p{.70\textwidth}|}
	\hline
	Use-Case & Wertpapier anlegen \\
	\hline
	Vorbedingungen & 
		\begin{itemize}
			\item Benutzer wurde angelegt
			\item Benutzer ist Angstellter
		\end{itemize} \\
	\hline
	Wesentliche Schritte &
		\begin{itemize}
			\item Mit Namen und Passwort anmelden
			\item Wertpapier anlegen
			\item Bestätigung erhalten
		\end{itemize} \\
	\hline
	Nachbedingungen & 
		\begin{itemize}
			\item Wertpapier wurde angelegt
			\item Aufträge für dieses Wertpapier können nun angelegt werden
		\end{itemize} \\
	\hline
\end{tabular}\par

\begin{tabular}{|p{.30\textwidth}|p{.70\textwidth}|}
	\hline
	Use-Case & Benutzer anlegen \\
	\hline
	Vorbedingungen & 
		\begin{itemize}
			\item Angestellter wurde angelegt
		\end{itemize} \\
	\hline
	Wesentliche Schritte &
		\begin{itemize}
			\item Mit Namen und Passwort anmelden
			\item Neuen Nutzer anlegen
			\item Bestätigung erhalten
		\end{itemize} \\
	\hline
	Nachbedingungen &
		\begin{itemize}
			\item Neuer Benutzer wurde angelegt
			\item NEuer Benutzer kann sich nun anmelden
		\end{itemize} \\
	\hline
\end{tabular}

\section{Domänenmodell}
\autoref{fig:domain} zeigt die wesentlichen Geschäftsobjekte, die auch von den
in \autoref{sec:use_case} beschriebene Anwendungsfällen genutzt werden.
Die Anwendungsfällen sind ein Teil der CRUD-Operationen angewendet auf den
Geschäftsobjekten.
\begin{figure}[H]
	\centering
    \includegraphics[width=\textwidth]{Domaenenmodell.png}
	\caption{Domänenmodell als Klassendiagramm}
	\label{fig:domain}
\end{figure}

\section{Komponentenmodel}
\autoref{fig:components} zeigt einen technische abstrakten Abriss der
Systemarchitektur. Es zeigt auch die Aufteilung der Komponenten in die
Schichten der 3-Schichten-Architektur. In Blau werden die externen Dienste
der Partnerprojekte \textit{investy} von Andreas Huber und \textit{Bank Amann}
von Sina Amann dargestellt.
\begin{figure}[H]
	\centering
    \includegraphics[width=\textwidth]{Komponentendiagramm.png}
	\caption{Systemarchitektur als Komponentenmodell}
	\label{fig:components}
\end{figure}

\subsection{Schnittstellenbeschreibungen}
\autoref{fig:interfaces} zeigt die Schnittstellenvereinbarungen für die Nutzung
der bereitgestellten Dienste. Die genutzten Datentypten Wertpapier und Auftrag
sind in \autoref{fig:domain} beschrieben. Die Schnittstellenvereinbarung für
den Überweisungsservice der \textit{Bank Amann} entnehmen Sie bitte dem
1. Meilenstein Dokument von Sina Amnann.
\label{subsec:Schnittstellen}
\begin{figure}[H]
	\centering
    \includegraphics[width=\textwidth]{Schnittstellen.png}
	\caption{Schnittstellenvereinbarungen als Interfaces}
	\label{fig:interfaces}
\end{figure}

\end{document}