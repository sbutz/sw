\documentclass[12pt, a4paper, titlepage]{article}

\usepackage{geometry}
\geometry{left=2.5cm, right=2.5cm, top=2.5cm, bottom=3cm}
\usepackage[utf8]{inputenc}
\usepackage[ngerman]{babel}
\usepackage{graphicx}
\usepackage{float}
\usepackage{parskip}
\usepackage[colorlinks, citecolor=black, filecolor=black, linkcolor=black,       
	urlcolor=black]{hyperref} 

\usepackage{booktabs}

\newcommand{\andi}[0]{Andreas Huber}
\newcommand{\andiProj}[0]{\textit{ynvest}}
\newcommand{\sina}[0]{Sina Amann}
\newcommand{\sinaProj}[0]{\textit{eBank}}


\begin{document}

\title{1. Meilenstein im Fach Softwareentwicklung}
\author{Stefan Butz \\
		Studiengruppen: IN7 \\
		Matrikelnummer: 3175600}

\maketitle

\section{Einleitung}
Dieses Dokument enthält die geforderten Bausteine für den ersten Meilenstein
im Fach Softwareentwicklung.
Als Softwareprojekt soll eine webbasierte Handelsplattform für Wertpapiere
realisiert werden.
Schnittstellen gibt es zu den Partnerprojekten \sinaProj{} von \sina{}
und \andiProj{} von \andi{}.
Eine genaue Beschreibung der Schnittstellen finden sich in
\autoref{subsec:Schnittstellen}.

\section{Anwendungsfälle}
\label{sec:use_case}
\subsection{Anwendungsfalldiagramm}
Die wesentlichen Funktionen, welche die Handelsplattform  bieten soll, werden in
\autoref{fig:use_case} dargestellt und im Folgenden genauer erläutert.

\begin{figure}[H]
	\centering
    %\includegraphics[width=\textwidth]{Use-Case-Diagramm.png}
    \includegraphics[height=15.5cm]{Use-Case-Diagramm.png}
	\caption{Use-Case-Diagramm mit wesentlichen Anwendungsfällen}
	\label{fig:use_case}
\end{figure}

\subsection{Use-Case-Beschreibungen}
\begin{tabular}{|p{.26\textwidth}|p{.69\textwidth}|}
	\hline
	Use-Case & Wertpapiere abfragen \\
	\hline
	Vorbedingungen &
		\begin{itemize}
			\item Wertpapiere wurden angelegt
		\end{itemize} \\
	\hline
	Wesentliche Schritte &
		\begin{itemize}
			\item Auflistung aller oder bestimmter Wertpapiere.
			Als Filter wird ein Wertpapierobjekt mit gesetzter ISIN verwendet
		\end{itemize} \\
	\hline
	Nachbedingungen & - \\
	\hline
\end{tabular}\par

\begin{tabular}{|p{.26\textwidth}|p{.69\textwidth}|}
	\hline
	Use-Case & Wertpapierdetails abfragen \\
	\hline
	Vorbedingungen &
		\begin{itemize}
			\item Wertpapier wurde angelegt
		\end{itemize} \\
	\hline
	Wesentliche Schritte &
		\begin{itemize}
			\item ISIN eines Wertpapieres angeben
			\item Details und Kurswerte dazu abfragen
		\end{itemize} \\
	\hline
	Nachbedingungen & - \\
	\hline
\end{tabular}\par

\begin{tabular}{|p{.26\textwidth}|p{.69\textwidth}|}
	\hline
	Use-Case & Auftrag erstellen \\
	\hline
	Vorbedingungen &
		\begin{itemize}
			\item Wertpapier, dass gehandelt werden soll, wurde angelegt
			\item Benutzer wurde angelegt
			\item Benuter hat sich mit Namen und Passwort oder API-Schlüssel
			authentifiziert
		\end{itemize} \\
	\hline
	Wesentliche Schritte &
		\begin{itemize}
			\item Auftrag ausfüllen (Welches Wertpapier soll in welcher Anzahl und zu
			welchem Stückpreis gekauft bzw. verkauft werden?)
			\item Auftrag abschicken
			\item Auftragsbestätigung mit Auftragnummer abrufen
		\end{itemize} \\
	\hline
	Nachbedingungen &
		\begin{itemize}
			\item Auftrag wurde gespeichert
			\item Sobald ein passender Gegenauftrag gefunden wird,
			wird der Auftrag abgeschlossen und eine Transaktion angelegt
			\item Geld wird über das Partnerprojekt \sinaProj{} auf das angegebene
			Konto überwiesen bzw. vom angegeben Konto eingezogen
			\item Falls eine BenachrichtungsUrl gesetzt wird der Nutzer nach
			Abschluss des Auftrags benachrichtigt.
			Das Partnerprojekt \andiProj{} bietet hier zu eine Schnittstelle an.
			\item Falls der Auftrag von einem Handelspartner erstellt wurde,
			wird die Auftragsgebühr von seinem Rechnungskonto über das
			Partnerprojekt \sinaProj{} eingezogen
		\end{itemize} \\
	\hline
\end{tabular}\par

\begin{tabular}{|p{.26\textwidth}|p{.69\textwidth}|}
	\hline
	Use-Case & Auftrag abfragen \\
	\hline
	Vorbedingungen &
		\begin{itemize}
			\item Benutzer wurde angelegt
			\item Benutzer hat Auftrag angelegt
			\item Benuter hat sich mit Namen und Passwort oder API-Schlüssel
			authentifiziert
		\end{itemize} \\
	\hline
	Wesentliche Schritte &
		\begin{itemize}
			\item Auftrag mithilfe der Auftragsnummer abfragen
			\item Auftrag mit aktuellem Status erhalten
		\end{itemize} \\
	\hline
	Nachbedingungen & - \\
	\hline
\end{tabular}\par

\begin{tabular}{|p{.26\textwidth}|p{.69\textwidth}|}
	\hline
	Use-Case & Wertpapier anlegen \\
	\hline
	Vorbedingungen & 
		\begin{itemize}
			\item Angestellter wurde angelegt
			\item Angestellter hat sich mit Namen und Passwort authentifiziert
		\end{itemize} \\
	\hline
	Wesentliche Schritte &
		\begin{itemize}
			\item Neues Wertpapierobjekt ausfüllen und abschicken
		\end{itemize} \\
	\hline
	Nachbedingungen & 
		\begin{itemize}
			\item Wertpapier wurde angelegt
			\item Aufträge für dieses Wertpapier können nun angelegt werden
		\end{itemize} \\
	\hline
\end{tabular}\par

\begin{tabular}{|p{.26\textwidth}|p{.69\textwidth}|}
	\hline
	Use-Case & Benutzer anlegen \\
	\hline
	Vorbedingungen & 
		\begin{itemize}
			\item Angestellter wurde angelegt
			\item Angestellter hat sich mit Namen und Passwort authentifiziert
		\end{itemize} \\
	\hline
	Wesentliche Schritte &
		\begin{itemize}
			\item Neues Benutzerobjekt ausfüllen und abschicken
		\end{itemize} \\
	\hline
	Nachbedingungen &
		\begin{itemize}
			\item Neuer Benutzer wurde angelegt
			\item Neuer Benutzer kann sich nun anmelden, Aufträge erstellen, etc.
		\end{itemize} \\
	\hline
\end{tabular}

\begin{tabular}{|p{.26\textwidth}|p{.69\textwidth}|}
	\hline
	Use-Case & Transaktionen abfragen \\
	\hline
	Vorbedingungen & 
		\begin{itemize}
			\item Angestellter wurde angelegt
			\item Angestellter hat sich mit Namen und Passwort authentifiziert
		\end{itemize} \\
	\hline
	Wesentliche Schritte &
		\begin{itemize}
			\item Transaktionen abfragen
		\end{itemize} \\
	\hline
	Nachbedingungen & - \\
	\hline
\end{tabular}

\section{Domänenmodell}
\autoref{fig:domain} zeigt die wesentlichen Geschäftsobjekte, die auch von den
in \autoref{sec:use_case} beschriebene Anwendungsfällen genutzt werden.
Die Anwendungsfällen sind ein Teil der CRUD-Operationen angewendet auf den
Geschäftsobjekten.
\begin{figure}[H]
	\centering
    \includegraphics[width=\textwidth]{Domaenenmodell.png}
    %\includegraphics[width=17cm]{Domaenenmodell.png}
	\caption{Domänenmodell als Klassendiagramm}
	\label{fig:domain}
\end{figure}

\section{Komponentenmodel}
\autoref{fig:components} zeigt einen technisch abstrakten Abriss der
Systemarchitektur. Es zeigt auch die Aufteilung der Komponenten in die
Schichten der 3-Schichten-Architektur. In rot werden die externen Dienste
der Partnerprojekte \andiProj{} von \andi{} und \sinaProj{}
von \sina{} dargestellt.
\begin{figure}[H]
	\centering
    \includegraphics[width=\textwidth]{Komponentendiagramm.png}
	\caption{Systemarchitektur als Komponentenmodell}
	\label{fig:components}
\end{figure}

\subsection{Schnittstellenbeschreibungen}
\autoref{fig:interfaces} zeigt die Schnittstellenvereinbarungen für die Nutzung
der bereitgestellten Dienste. Die genutzten Datentypten Wertpapier und Auftrag
sind in \autoref{fig:domain} beschrieben. Die Schnittstellenvereinbarung für
den Überweisungsservice von \sinaProj{} und den AuftragService von \andiProj{}
entnehmen Sie bitte dem 1. Meilenstein Dokument von \sina{} und \andi{}.
\label{subsec:Schnittstellen}
\begin{figure}[H]
	\centering
    \includegraphics[width=\textwidth]{Schnittstellen.png}
	\caption{Schnittstellenvereinbarungen als Interfaces}
	\label{fig:interfaces}
\end{figure}

\end{document}